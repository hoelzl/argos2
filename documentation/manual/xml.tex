\lstset{
  language=XML,
  basicstyle=\footnotesize,
  frame=leftline,
  xleftmargin=1em,
  framexleftmargin=0.3em,
  framerule=2pt
}

\chapter{The XML Configuration File}
To run a simulation, the user must provide a correct XML configuration
file. Such file serves diverse purposes: describing the static layout
of the arena, placing the robots, declaring which controllers to use,
and more. In the following we will see how this file is organized.

\section{The General Structure}
The XML configuration file of \argos is composed of 5 required section
and 2 optional ones:
\begin{description}
\item[\texttt{framework} (req):] here you set some internal parameters of \argos;
\item[\texttt{controllers} (req):] a list of the controllers to use in the
  experiment, along with a specification of their configuration;
\item[\texttt{arena} (req):] contains the objects that populate the simulated
  space (i.e., their position, orientation, ...);
\item[\texttt{physics\_engines} (req):] a list of the physics engines to use in
  the environment;
\item[\texttt{arena\_physics} (req):] the mapping between objects in the
  simulated space and the physics engines. Mobile entities can only
  belong to one physics engine, while immobile entities usually belong
  to multiple engines;
\item[\texttt{visualization} (opt):] if this section is present, the selected
  visualization is loaded. Otherwise, \argos runs silently.
\item[\texttt{loop\_functions} (opt):] if this section is present, the
  user-defined that contains the dynamic aspects of the experiment is
  loaded.
\end{description}

\subsection{The \texttt{framework} Section}
This section has two tags: \texttt{system} and
\texttt{experiment}. Let's see an example:
\begin{lstlisting}
<framework>
  <system threads="2" />
  <experiment length="70"
              ticks_per_second="10"
              random_seed="124" />
</framework>
\end{lstlisting}
The \texttt{system} tag currently contains only one attribute,
\texttt{threads}, that stores the number of slave threads used by
\argos for the computation. If you set it to zero, no threads are
employed. If you set it to a number $N > 0$, $N$ slave threads are
used. Performance evaluation shows that it makes sense to use threads
only when the number of robots to simulate is larger than about
100. Below this threshold, the cost of managing the threads exceeds
the benefits. In case you want to use threads, for maximum performance
it is better to set the number of threads equal to the number of cores
present in your system.

The \texttt{experiment} section is used to specify some general
aspects of the simulation. The \texttt{length} attribute contains the
desired lenght of the experiments in seconds. If you set it to zero,
the experiment runs infinitely, or until another custom ending
condition is met as specified in the \texttt{IsExperimentFinished()}
method in the loop functions (for more information, see
Chapter~\ref{cpt:loop_functions}). The \texttt{ticks\_per\_second}
attribute specifies how many control steps (ticks) are executed per
second. This value is usually set to 10, which corresponds to
100ms-long control steps. Finally, if you set the
\texttt{random\_seed} attribute, the simulation will be
repeatable. Noise will be calculated on the basis of the random seed
you set. If you don't set it or choose zero as value, the internal
computer clock is used to find a random seed and a warning is
displayed at run-time.

\subsection{The \texttt{controllers} Section}
In the \texttt{controllers} section you specify the robot controllers
to use for the experiment:
\begin{lstlisting}
<controllers>

  <my_controller_1 id="c1" library="/path/to/libmy_controller_1.so">
    <actuators>
      ...
    </actuators>
    <sensors>
      ...
    </sensors>
    <parameters>
      ...
    </parameters>
  </my_controller_1>

  <my_controller_2 id="c2" library="/path/to/libmy_controller_2.so">
    ...
  </my_controller_2>

</controllers>
\end{lstlisting}
In Chapter~\ref{cpt:controllers} we said that controllers are
registered in the system through the \texttt{REGISTER\_CONTROLLER()}
macro. The string identifier you specify as parameter to the macro is used
in the XML file as tag (here, \texttt{my\_controller\_1} and
\texttt{my\_controller\_2}).

The configuration of each controller is performed specifying the
following information:
\begin{itemize}
\item a unique identifier, used in the \texttt{arena} section to bind
  robots to controllers (attribute \texttt{id});
\item the complete path to the plug-in library file (attribute
  \texttt{library});
\item which actuators and sensors to use, along with their
  configuration. As shown in the above example, a sub-section is
  dedicated to each (\texttt{actuators} and \texttt{sensors});
\item each robot controller accepts an XML sub-tree that is (optionally)
  parsed by the \texttt{Init()} method. The passed tree is the
  \texttt{parameters} sub-section.
\end{itemize}

\subsection{The \texttt{arena} Section}
\subsection{The \texttt{physics\_engines} Section}
\subsection{The \texttt{arena\_physics} Section}
\subsection{The \texttt{visualization} Section}
\subsection{The \texttt{loop\_functions} Section}

%%% Local Variables: 
%%% mode: latex
%%% TeX-master: "argos_user_manual"
%%% End: 
